\documentclass[14pt, oneside]{extreport}
\usepackage{setspace}
\setstretch{1.15} % Modify the value to adjust the line spacing
\usepackage[a4paper, margin=3.3cm]{geometry}
\usepackage[english]{babel}
\usepackage{microtype}
\usepackage{lipsum}
\usepackage{fancyhdr}
\usepackage[style=numeric]{biblatex}
\usepackage{csquotes}
\usepackage{graphicx}
\usepackage{listings}
\usepackage{xcolor}
\graphicspath{ {./images/} }

\definecolor{codegreen}{rgb}{0,0.6,0}
\definecolor{codegray}{rgb}{0.5,0.5,0.5}
\definecolor{codepurple}{rgb}{0.58,0,0.82}
\definecolor{backcolour}{rgb}{0.95,0.95,0.92}

\lstdefinestyle{mystyle}{
    backgroundcolor=\color{backcolour},   
    commentstyle=\color{codegreen},
    keywordstyle=\color{magenta},
    numberstyle=\tiny\color{codegray},
    stringstyle=\color{codepurple},
    basicstyle=\ttfamily\footnotesize,
    breakatwhitespace=false,         
    breaklines=true,                 
    captionpos=b,                    
    keepspaces=true,                 
    numbers=left,                    
    numbersep=5pt,                  
    showspaces=false,                
    showstringspaces=false,
    showtabs=false,                  
    tabsize=2
}

\lstset{style=mystyle}
\addbibresource{references.bib}
\fancyhead[LE, RO]{\thepage}
\fancyhead[RE]{\leftmark}
\fancyhead[LO]{\rightmark}
%\fancyfoot[LE, RO]{\thepage}
\fancyfoot[CO, CE]{}

% STEP 1: include the package
\usepackage{uninafrontespizio}
% STEP 2: Front-page configuration
\Universita{Università degli Studi di Napoli Federico II}
\Facolta{Scuola Politecnica e delle Scienze di Base}
\Dipartimento{Dipartimento di Ingegneria Industriale}
\CorsoDiLaurea{Corso di Laurea in Ingegneria Aerospaziale\\
Classe delle Lauree in Ingegneria Industriale (L-9)}
\Materia{Elaborato di Laurea in Meccanica del Volo} % optional
\AnnoAccademico{Anno Accademico 2023--2024}
\Titolo{Titolo Tesi}
\Relatore{Prof.\ Danilo \textsc{Ciliberti}}
\RelatoreLabel{Relatore} %optional, default: Relatore
%\relandcorrelsep{2em} %optional, vertical space between relatori and correlatori default: 1.5ex
\Correlatore{[Rimuovi se non necessario]} % Cancella l'argomento se non esiste il correlatore
%\Correlatore{Dott.Foo \textsc{Bar}} % can add as many "Correlatori" as you wish
%\CorrelatoreLabel{Controrelatore} %optional, default: Correlatore
\Candidato{Nome \textsc{Cognome}} %only one candidate is currently supported
%\CandidatoLabel{Candidata} %optional, default: Candidato
\Matricola{N35/0000}
\Logo{logo-federico-II.pdf} %path to logo image
\LogoWidth{4cm} %optional, default: 3cm
\LogoPosition{below-uni} % or top, or below-title, or above-title, or no-logo




\begin{document}
    % STEP 3: use \makefrontpage and/or \makefrontpagealt
    
\begin{titlepage}

\pagestyle{empty}
\makefrontpage

\end{titlepage}
    \nocite{*}
    \thispagestyle{empty}
\begin{center}
    \vspace*{\fill}
    \raggedleft % Aligns the dedication to the right
    \itshape
    
    Dedication sentence
    
    \vspace*{\fill}
\end{center}
    \newpage
    \section*{Abstract}
Your abstract in English goes here.

\section*{Sommario}
\begin{otherlanguage}{italian}
Il tuo sommario in italiano va qui.
\end{otherlanguage}
    \pagestyle{empty}
    
    \tableofcontents
    \pagestyle{fancy}

    \chapter{Introduction}
    % Chapter 1 - Introduction
\lipsum[1-3]
    \chapter{Methods}
    % Chapter 2 - Methods
\lipsum[4-6]
    \chapter{Results}
    % Chapter 3 - Results
\lipsum[7-9]
    \chapter{Conclusions}
    % Chapter 4 - Conclusions
\lipsum[10-12]
    \chapter*{Appendix}
    This chapter is optional. If more than one appendix chapter is needed, number it as Appendix A, Appendix B, and so on.
    \chapter*{Ringraziamenti}
\begin{otherlanguage}{italian}
Qui vanno i ringraziamenti, se desideri metterli. Questo capitolo, come l'appendice, è facoltativo.
\end{otherlanguage}
    \printbibliography
\end{document}